%\documentclass[]{article}
\documentclass[11pt,onecolumn]{article}
\usepackage{pset}

\title{MATH 4500 - Homework 2}
\author{Daniel Alonso Vicuna}
\date{February 6, 2019}

\begin{document}
\maketitle

\begin{exercise}
Show that there is exactly one n-element subgroup of $SO(2)$, for each natural number $n$, and list its members. 
\end{exercise}
\begin{answer}

First we show that if $G$ is a finite group of order $n$, then for any $g \in G$ it must be the case that $g^n = e$. Suppose, for the sake of contradiction, that this is not the case. First, it cannot be the case that $g^m \neq e$ for $0 \leq m \leq m$, otherwise $\{e,g,g^2,...,g_n\}$ are all distinct and the order of $G$ must be greater than $n$ - a contradiction. Hence, suppose that $g^m = e$ for some $0 \leq m \leq n-1$. We can define the group generated by $g$, that is $\langle g \rangle = \{ g^k \mid k \in \z\}$. By Lagranges theorem, $|\langle g \rangle|$ divides $|G|$ and hence, we can write $$g^n = g^{|\langle g\rangle|l} = (g^{|\langle g\rangle|})^l = e^l = e \quad l \in \z$$

Next we show that $\s^1 \cong SO(2)$ by using the obvious isomorphism $\varphi: \s^1 \rightarrow SO(2)$:
$$ \varphi(e^{i\phi}) = \begin{pmatrix} \cos(\phi) & -\sin(\phi) \\ \sin(\phi) & \cos(\phi) \end{pmatrix} $$. We can check that the group structure is preserved since
\begin{align*}
\varphi(e^{i\phi_1}e^{i\phi_2}) = \varphi(e^{i(\phi_1+\phi_2)}) &=  \begin{pmatrix} \cos(\phi_1 + \phi_2) & -\sin(\phi_1 + \phi_2) \\ \sin(\phi_1 + \phi_2) & \cos(\phi_1 + \phi_2) \end{pmatrix} \\
&=\begin{pmatrix} \cos(\phi_1 ) & -\sin(\phi_1) \\ \sin(\phi_1) & \cos(\phi_1) \end{pmatrix} \begin{pmatrix} \cos(\phi_2) & -\sin(\phi_2) \\ \sin( \phi_2) & \cos( \phi_2) \end{pmatrix}\\
&= \varphi(e^{i\phi_1})\varphi(e^{i\phi_2})
\end{align*}

So now we look at elements in $\s^1$, and let H be a subgroup of $SO(2)$ with $n$ elements. From the proof above, it follows that $g^n = e $ for all $g \in H$. Notice that the only members of $\s^1$ that satisfy this are
$$S = \{ e^\frac{2i\pi k}{n} \mid k=0,...,n-1\}$$
Since $|S| = n$ and $|H|= n$, then it follows that $H$ is uniquely $H  = \{ \varphi(x) \mid x \in S\}$.


\end{answer}
\begin{exercise}
 Show that ${\pm \one}$ is a normal subgroup of $\s^3$.
\end{exercise}
\begin{answer}
Take any $g \in \s^3$. Then we can show that $g\inv \{\pm \one \} g = \{\pm \one\}$. This holds because for any $g = a\one + b \ii + c\jj + d\kk \in \s^3$ and $x \in \real$, we have that $xg = x(a\one + b \ii + c\jj + d\kk) = xa\one + xb \ii + xc\jj + xd\kk = a\one x + b \ii x + c\jj x + d\kk x =  gx$, and thus $$g\inv\{\pm \one\} g = \{\pm \one \} g\inv g = \{\pm \one \}$$

Thus, we conclude that $\{ \pm \one\} $ is normal subgroup of $\s^3$

\end{answer}



\begin{exercise}
 Show that $\s^1$ is not a normal subgroup of $\s^3$.
\end{exercise}
\begin{answer}

In order to show that $\s^1$ is not a normal subgroup of $\s^3$, we can find a $g \in S^3$ such that $g\inv S^1 g \neq S^1$. One such $g$ is the quaternion $$g\inv = \frac{1}{\sqrt{2}}(2\ii + 2\kk) \quad g = \frac{1}{\sqrt{2}}(-2\ii - 2\kk)$$
and we can compute $g\inv \s^1 g$:
\begin{align*}
    g\inv \s^1 g &= \{\frac{1}{2}(-\ii -\kk)(\cos(\theta) + \ii \sin(\theta))(\ii + \kk) \mid \theta \in \real \} \\
    &= \{
    \frac{1}{2}( -\ii \cos(\theta) + \sin(\theta) -\kk \cos(\theta) - \jj \sin(\theta)  ) (\ii + \kk)
    \mid \theta \in \real \} \\
    &= \{
    \frac{1}{2}(2\cos(\theta) + 2\kk \sin(\theta) )  \mid \theta \in \real  \} \\
    &= \{\cos(\theta) + \kk \sin(\theta) \mid \theta \in \real
    \}
    \neq \s^1
\end{align*}
This shows that $\s^1$ is indeed not a normal subgroup of $\s^3$.
\end{answer}
\begin{exercise}
Show that reflection in the hyper-plane orthogonal to a coordinate axis has determinant -1, and generalize this result to any reflection.
\end{exercise}
\begin{answer}
Let $e_1,...,e_n$ be some basis vectors for $\real^n$. Consider, without loss of generality, a reflection in the hyperplane $H$ orthogonal to $e_1$. The reflection is a linear map that fixes all points in $H$ and reverses the first coordinate, and can thus be represented by the matrix
$$R_1 =  \begin{pmatrix} -1 & 0 \\ 0 & \one \end{pmatrix},$$
where $\one \in SO(n-1)$. Clearly, this matrix has determinant -1. We can generalize this to reflections across any hyperplane by considering the unit vector $\vec{u}_1$ orthogonal to the hyperplane, and extending that to an orthonormal basis $u_2,...,u_n$, then composing the change of basis matrix
$$ U = \begin{pmatrix} u_1 & .... & u_n \end{pmatrix}. $$ Since all of $u_1,...,u_n$ have norm 1 and are orthogonal, $U^TU = \one$ and so $U$ has determinant $\pm 1$, and the reflection map is actually given by $U^TR_1U$. Clearly,
$$ det(U^TR_1U) = det(U^T)det(U)det(R_1) = det(U^TU)det(R_1) = 1 * -1 = -1$$

\end{answer}



\begin{exercise}
 Observe that the rotations in Exercise 2.6.1 form an $S^1$, as do the rotations in Exercise 2.6.2, and deduce that $SO(4)$ contains a subgroup isomorphic to $T^2$. 
\end{exercise}
\begin{answer}

The matrices of Exercise 2.6.1 have the form $$ \begin{pmatrix} R_\theta & 0 \\ 0 & \one \end{pmatrix} $$ while matrices in Exercise 2.6.2 have the form $$ \begin{pmatrix} \one & 0 \\ 0 & R_\phi \end{pmatrix} $$ where $\one, R_\phi, R_\theta \in SO(2)$. For this reason, we can write all such matrices, in general, in the form $$ \begin{pmatrix} R_\theta & 0 \\ 0 & R_\phi \end{pmatrix} $$

I claim that the union of these matrices, which we denote by $G$, is a subgroup of $SO(4)$ (under matrix multiplication). We check all the necessary properties:

\begin{itemize}
    \item Clearly, since $\one_{2 \times 2} \in SO(2)$, it follows that $\one \in G$.
    \item Any such matrix has an inverse and is contained in the set, and given by $-\theta$ and $-\phi$ respectively.
    \item The set is closed under multiplication, since these are block matrices and $SO(2)$ is itself a group.
    \item Matrix multiplication is associative.
\end{itemize}

The set of these matrices is isomorphic to the group $SO(2) \times SO(2)$ by page 41 of the textbook, and $SO(2)$ we recall from question 1 that $\s^1 \cong SO(2)$ so $\s^1 \times \s^1 \cong SO(2) \times SO(2)$ and hence $G \subset SO(4)$ is a subgroup and it is isomorphic to $\s^1 \times \s^1 = \mathbf{T}^2$.

\end{answer}
\begin{exercise}
Explain why $S^3 = SU(2)$ is not the same group as $S^1 \times S^1\times S^1$. 
\end{exercise}
\begin{answer}
The notation is misleading: $\s^1 \times \s^1 \times \s^1$ is the 3-fold direct product of $\s$ with itself, while $\s^3 = SU(2)$ is the group of unit quaternions. One way to check that $\s^1 \times \s^1 \times \s^1 \ncong SU(2)$ is to check for invariants in both groups. For example, the number of holes in $\s^1 \times \s^1 \times \s^1$, also or $T^3$ is 3, while $SU(2)$ has 0 holes. Hence, $SU(2) \ncong T^3$.

\end{answer}
\end{document}
