\documentclass[12pt,onecolumn]{article}
\usepackage{pset}

\title{MATH 4500 - Homework 7}
\author{Daniel Alonso Vicuna}
\date{March 13, 2019}

\begin{document}
\maketitle

\begin{exercise}
Using bilinearity, or otherwise, show that $U,V \in \real \ii+ \real \jj+ \real \kk$ implies $[U,V] \in  \real \ii+ \real \jj+ \real \kk$.
\end{exercise}
\begin{answer}
We show this using the definition of the Lie bracket. First, we compute the values of several Lie brackets that will be useful:
\begin{align*}
    [i,j] &= ij - ji = 2k \\
    [j,k] &= jk - kj = 2i \\
    [k,i] &= ik - ki = 2j 
\end{align*}
Notice that, just like with the cross product in $\real^3$, we can write this more succintly using the antisymemtric levi-civita tensor, if we let $e_1=\ii, e_2=\jj, e_3=\kk$:
\begin{align*}
    [e_i, e_j] = 2 e_k \epsilon_{ijk}
\end{align*}
Now we let $U = u_1\ii + u_2\jj + u_3\kk = \sum u_l e_l$ and $V = v_1\ii + v_2\jj + v_3\kk = \sum v_l e_l$, and compute $[U,V]$:
\begin{alignat*}{2}
    [\sum_l u_l e_l, \sum_m v_m e_m] &= \sum_l \sum_m u_l v_m [e_l,e_m] \quad &\text{ bilinear} \\
    &= \sum_j \sum_m u_l v_m e_k \epsilon_{lmk} \quad &\text{ from above}
\end{alignat*}
But every term in the sum is in $\real\ii + \real\jj + \real\kk$, and hence, so is $[U,V]$.
\end{answer}
\begin{exercise}
Prove the Jacobi identity by using the definition $[X,Y]=XY-YX$ to expand $[X,[Y,Z]]+[Y,[Z,X]]+[Z,[X,Y]]$. Assume only that the product is associative and that the usual laws for plus and minus apply. 
\end{exercise}
\begin{answer}
We use the definition of the Lie bracket. First we look at the term $[X,[Y,Z]]$:
\begin{align*}
    [X,[Y,Z]] &= [X,YZ - ZY] \quad \text{ by defn } \\
    &= [X,YZ] - [X,ZY] \quad \text{ bilinear } \\
    &= XYZ - YZX - XZY + ZYX
\end{align*}
Now we plug this in to the complete expression, noting that each term is a cyclic permutation of the previous one:
\begin{align*}
    [X,[Y,Z]]+[Y,[Z,X]]+[Z,[X,Y]] &= XYZ - YZX - XZY + ZYX \\
    &+ YZX - ZXY - ZYX + YXZ \\
    &+ ZXY - XYZ - YXZ + XZY \\
    &= 0
\end{align*}
\end{answer}
\begin{exercise} We explore the correspondence between $SO(3)$ and skew-symmetric matrices.
\begin{enumerate}[label=(\alph*)]
    \item Find the exponential of the matrix $B = \begin{pmatrix} 0 & -\theta & 0 \\ \theta & 0 & 0 \\ 0 & 0 & 0 \end{pmatrix} = \begin{pmatrix} \theta J & 0 \\ 0 & 0 \end{pmatrix}$
    \begin{answer}
    Using block multiplication, this will converge to $$ e^B = \begin{pmatrix} e^{\theta J} & 0 \\ 0 & e^0 \end{pmatrix}$$
    with $J$ defined as in Homework 5, Exercise 5. But as we found in that same exercise
    $$ e^{\theta J} = \begin{pmatrix} \cos{\theta} & -\sin{\theta} \\ \sin{\theta} & \cos{\theta} \end{pmatrix}.$$
    It follows that $$e^B = 
    \begin{pmatrix} 
    \cos{\theta} & -\sin{\theta} & 0 \\
    \sin{\theta} & \cos{\theta} & 0\\ 
    0 & 0 & 1 
    \end{pmatrix}$$
    \end{answer}
    \item Show that $Ae^BA^T =e^{ABA^T}$ for any orthogonal matrix $A$. 
    \begin{answer}
    First we prove by induction on $n$ that for any matrix $A \in SO(3)$, $(ABA^T)^n = AB^nA^T$. The base case $n=1$ trivially holds, and we assume it holds up to $n$ and show that its true for $n+1$:
    \begin{align*}
        (ABA^T)^{n+1} &= (ABA^T)(ABA^T)^{n} \\
        &= (ABA^T)(AB^nA^T) \quad \text{ induction }\\
        &= AB(A^TA)B^nA^T \\
        &= AB^{n+1}A^T
    \end{align*}
    Hence, we can use the definition of the matrix exponential to conclude the exercise:
    \begin{align*}
        e^{ABA^T} &= \one + \sum_{n=1} (ABA^T)^n \\
        &= \one + \sum_{n=1} AB^nA^T \\
        &= A(A^T + \sum_{n=1} B^nA^T ) \\
        &= A(\one + \sum_{n=1} B^n)A^T = Ae^BA^T
    \end{align*}
    \end{answer}
    \item Deduce that each matrix in $SO(3)$ equals $e^X$ for some skew-symmetric $X$.
    \begin{answer}
    First note that any matrix in $Y \in SO(3)$, since it is a rotation of $\real^3$, can be written as $AR_z(\theta)A^T$ for some orthogonal matrix  $A$, with $R_z(\theta) \in SO(3)$ being the rotation on the z-axis (this is just a change of basis). But from part (a), we found that $e^B = R_z(\theta)$, and hence $Ae^BA^T = Y$. Using part (b), we find that $Ae^BA^T = e^{ABA^T} = Y$ and hence any matrix $Y \in SO(3)$ equals $e^{ABA^T}$ for some $A \in O(3)$. 
    
    Finally, we show that $ABA^T$ is skew-symmetric:
    \begin{align*}
        ABA^T + (ABA^T)^T &= ABA^T + AB^TA^T \\
        &= A(B+B^T)A^T \\
        &= A(B-B)A^T = 0
    \end{align*}
    \end{answer}
\end{enumerate}

\end{exercise}

\begin{exercise}
Show that the skew-Hermitian matrices in the tangent space of $SU(2)$ can be written in the form $b\ii+c\jj+d\kk$, where $b,c,d \in \real$ and $\ii, \jj$, and $\kk$ are matrices with the same multiplication table as the quaternions $\ii, \jj$, and $\kk$.
\end{exercise}
\begin{answer}
By the proof about the tangent space of $SU(n)$ in Section 5.3 , any $X \in SU(n)$ must satisfy
$$  X + \overline{X}^T = 0 \quad \text{and} \quad Tr(X) = 0$$

Clearly, any such matrix $X$ must be of the form:
\begin{align*}
    X &= \begin{pmatrix} d\ii & -b + -c\ii \\ b + -c\ii & -d\ii \end{pmatrix} \\
    &= b \begin{pmatrix} 0 & -1 \\ 1 & 0 \end{pmatrix} + c \begin{pmatrix} 0 & -\ii \\ -\ii & 0 \end{pmatrix} + d \begin{pmatrix} \ii & 0 \\ 0 & -\ii \end{pmatrix}
\end{align*}
for $b,c,d \in \real$ where $\ii,\jj,\kk$. But these matrices are the ones we saw in page 9 of the textbook, where we saw that the indeed satisfy the multiplication table of the quaternions $\ii, \jj, \kk$.
\end{answer}
\begin{exercise}
Also find the tangent space of $Sp(1)$ (which should be the same)
\end{exercise}
\begin{answer}
Once again, we have the equation for any $q$ in the Tangent space of $Sp(1)$.
$$ q + \overline{q}^T = 0$$
But we can decompose this in terms of the usual $1, \ii,\jj,\kk$:
\begin{align*}
    (a + b\ii + c\jj + d\kk) + \overline{a + b\ii + c\jj + d\kk} &= 0 \\
    (a+b\ii + c\jj +d\kk) + (a-b\ii -c\jj-d\kk) &= 0\\
    \implies 2a &= 0\\
\end{align*}
Hence, the tangent space any linear combination of $\ii,\jj,\kk$. In addition these $\ii,\jj,\kk \in \mathbb{H}$ trivially satisfy the multiplication table of quaternions.
\end{answer}
\begin{exercise}
Interpret the following sum as $Tr(XY)$ and $Tr(YX)$:
\begin{align*}
      x_{11}y_{11} +  x_{12}y_{21} + ... +  x_{1n}y_{n1}\\
    + x_{21}y_{12} +  x_{22}y_{22} + ... +  x_{2n}y_{n2}\\
    \vdots \quad \\
    + x_{n1}y_{1n} +  x_{n2}y_{2n} + ... +  x_{2n}y_{n2}
\end{align*}
\end{exercise}
\begin{answer}
We have
\begin{align*}
    Tr(XY) &= Tr(\sum_j X_{ij}Y_{jk}) \\
    &= \sum_{k=1}^n \delta_{ik}\sum_{j=1}^n X_{ij}Y_{jk} \\
    &= \sum_{k=1}^n\sum_{j=1}^n  X_{kj}Y_{jk}
\end{align*}
Clearly, for any $k$, we are summing over a row in the expression above. Similarly,
\begin{align*}
    Tr(YX) &= Tr(\sum_j Y_{ij}X_{jk}) \\
    &= \sum_{k=1}^n \delta_{ik}\sum_{j=1}^n Y_{ij}X_{jk} \\
    &= \sum_{k=1}^n\sum_{j=1}^n  Y_{kj}X_{jk}
\end{align*}
and so for any $k$, we are summing over a column in the expression above. It follows that $Tr(XY)=Tr(YX)$, as desired.
\end{answer}

\end{document}
