\documentclass[12pt,onecolumn]{article}
\usepackage{pset}

\title{MATH 4500 - Homework 6}
\author{Daniel Alonso Vicuna}

\begin{document}
\maketitle

\begin{exercise}
Using bilinearity, or otherwise, show that $U,V \in \real \ii+ \real \jj+ \real \kk$ implies $[U,V] \in  \real \ii+ \real \jj+ \real \kk$.
\end{exercise}
\begin{answer}
We show this using the definition of the Lie bracket. First, we compute the values of several Lie brackets that will be useful:
\begin{align*}
    [i,j] &= ij - ji = 2k \\
    [j,k] &= jk - kj = 2i \\
    [k,i] &= ik - ki = 2j 
\end{align*}
Notice that, just like with the cross product in $\real^3$, we can write this more succintly using the antisymemtric levi-civita tensor, if we let $e_1=\ii, e_2=\jj, e_3=\kk$:
\begin{align*}
    [e_i, e_j] = 2 e_k \epsilon_{ijk}
\end{align*}
Now we let $U = u_1\ii + u_2\jj + u_3\kk = \sum u_l e_l$ and $V = v_1\ii + v_2\jj + v_3\kk = \sum v_l e_l$, and compute $[U,V]$:
\begin{alignat*}{2}
    [\sum_l u_l e_l, \sum_m v_m e_m] &= \sum_l \sum_m u_l v_m [e_l,e_m] \quad &\text{ bilinear} \\
    &= \sum_j \sum_m u_l v_m e_k \epsilon_{lmk} \quad &\text{ from above}
\end{alignat*}
But every term in the sum is in $\real\ii + \real\jj + \real\kk$, and hence, so does $[U,V]$.
\end{answer}
\begin{exercise}
Prove the Jacobi identity by using the definition $[X,Y]=XY-YX$ to expand $[X,[Y,Z]]+[Y,[Z,X]]+[Z,[X,Y]]$. Assume only that the product is associative and that the usual laws for plus and minus apply. 
\end{exercise}
\begin{answer}
We use the definition of the Lie bracket. First we look at the term $[X,[Y,Z]]$:
\begin{align*}
    [X,[Y,Z]] &= [X,YZ - ZY] \quad \text{ by defn } \\
    &= [X,YZ] - [X,ZY] \quad \text{ bilinear } \\
    &= XYZ - YZX - XZY + ZYX
\end{align*}
Now we plug this in to the complete expression, noting that each term is a cyclic permutation of the previous one:
\begin{align*}
    [X,[Y,Z]]+[Y,[Z,X]]+[Z,[X,Y]] &= XYZ - YZX - XZY + ZYX \\
    &+ YZX - ZXY - ZYX + YXZ \\
    &+ ZXY - XYZ - YXZ + XZY \\
    &= 0
\end{align*}
\end{answer}
\begin{exercise}
$$ B = \begin{pmatrix} 0 & -\theta & 0 \\ \theta & 0 & 0 \\ 0 & 0 & 0 \end{pmatrix} \implies e^B = \begin{pmatrix} \cos{\theta} & -\sin{\theta} & 0 \\ \sin{\theta} & \cos{\theta} & 0 \\ 0 & 0 & 1 \end{pmatrix}  $$
Using the fact above, and that for any orthogonal matrix $A$,  $Ae^{B}A^T =e^{ABA^T}$. Deduce that each matrix in $SO(3)$ equals $e^X$ for some skew-symmetric matrix $X$.
\end{exercise}
\begin{answer}
From block multiplication and a previous exercise, we know that the implication in the question holds. Furthermore, we know that any matrix $A \in SO(3)$ has the property $AA^T = e$. 
\end{answer}
\begin{exercise}
Show that the skew-Hermitian matrices in the tangent space of $SU(2)$ can be written in the form $b\ii+c\jj+d\kk$, where $b,c,d \in \real$ and $\ii, \jj$, and $\kk$ are matrices with the same multiplication table as the quaternions $\ii, \jj$, and $\kk$.
\end{exercise}
\begin{answer}
First note that any vector $X$ in the tangent space of $SU(2)$ satisfies the equation
$$ X + \overline{X}^T = 0$$
Clearly, any such matrix $X$ must be of the form:
\begin{align*}
    X &= \begin{pmatrix} a\ii & -b + -c\ii \\ b + -c\ii & -a\ii \end{pmatrix} \\
    &= b \ii + c \jj  + a \kk 
\end{align*}
for $b,c,d \in \real$ where $\ii,\jj,\kk$ are defined as in page 9 of the textbook.
\end{answer}
\begin{exercise}
Also find the tangent space of $Sp(1)$ (which should be the same)
\end{exercise}
\begin{answer}
Once again, we have the equation (for a $1\times 1$ matrix)
$$ q + \overline{q}^T = 0$$
But we can decompose this in terms of the usual $\ii,\jj,\kk$ since this is a quaternion. Furtermore, since $a + \overline{a} = 0 $ implies $a =0$ if $a \in \real$, then it follows that the equation above takes the form:
\begin{align*}
    b\ii + c\jj + d\kk + \overline{b\ii + c\jj + d\kk}^T &= 0 \\
    b \ii + c\jj + d \kk -b\ii -c\jj -d\kk &0
\end{align*}
In addition these $\ii,\jj,\kk \in \mathbb{H}$ trivially satisfy the multiplication table of quaternions.
\end{answer}
\begin{exercise}
Interpret the following sum as $Tr(XY)$ and $Tr(YX)$:
\begin{align*}
      x_{11}y_{11} +  x_{12}y_{21} + ... +  x_{1n}y_{n1}\\
    + x_{21}y_{12} +  x_{22}y_{22} + ... +  x_{2n}y_{n2}\\
    \vdots \quad \\
    + x_{n1}y_{1n} +  x_{n2}y_{2n} + ... +  x_{2n}y_{n2}
\end{align*}
\end{exercise}
\begin{answer}
We have
\begin{align*}
    Tr(XY) &= Tr(\sum_j X_{ij}Y_{jk}) \\
    &= \sum_{k=1}^n \delta_{ik}\sum_{j=1}^n X_{ij}Y_{jk} \\
    &= \sum_{k=1}^n\sum_{j=1}^n  X_{kj}Y_{jk}
\end{align*}
Clearly, for any $k$, we are summing over a row in the expression above. Similarly,
\begin{align*}
    Tr(YX) &= Tr(\sum_j Y_{ij}X_{jk}) \\
    &= \sum_{k=1}^n \delta_{ik}\sum_{j=1}^n Y_{ij}X_{jk} \\
    &= \sum_{k=1}^n\sum_{j=1}^n  Y_{kj}X_{jk}
\end{align*}
and so for any $k$, we are summing over a column in the expression above.
\end{answer}

\end{document}
